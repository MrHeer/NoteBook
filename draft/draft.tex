\documentclass[12pt]{article}

\usepackage{amsmath}
\usepackage[utf8]{inputenc}
\usepackage{hyperref}

\hypersetup{
colorlinks=true,
linkcolor=black
}
\numberwithin{equation}{section}

\title{The Draft of MrHeer}
\author{MrHeer}
\date{\today}
 
\begin{document}

\begin{titlepage}
    \maketitle
    \thispagestyle{empty}
\end{titlepage}

\begin{center}
    \tableofcontents
\end{center}
\pagenumbering{roman}

\newpage
\pagenumbering{arabic}
\setcounter{page}{1}
\section{Pythagorean}
The Pythagorean theorem is:
\begin{equation}
    a^2 + b^2 = c^2 \label{pythagorean}
\end{equation}
Equation \eqref{pythagorean} is called ‘Gougu theorem’ in Chinese.

\end{document}
